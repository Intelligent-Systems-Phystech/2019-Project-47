\documentclass[12pt,twoside]{article}
\usepackage{jmlda, amssymb, amsmath}
%\NOREVIEWERNOTES
    \title
    [Сферические свёрточные нейронные сети для QSAR предсказаний] % Краткое название; не нужно, если полное название влезает в~колонтитул
    {Сферические свёрточные нейронные сети \newline для  QSAR предсказаний}
    \author
    [Автор~И.\,О.] % список авторов для колонтитула; не нужен, если основной список влезает в колонтитул
    {Автор~И.\,О., Соавтор~И.\,О., Фамилия~И.\,О.} % основной список авторов, выводимый в оглавление
    [Вареник Н.В., Попова М.С., Стрижов В.В] % список авторов, выводимый в заголовок; не нужен, если он не отличается от основного
    \thanks
    {Работа выполнена при финансовой поддержке РФФИ, проект \No\,00-00-00000.
    	Научный руководитель:  Стрижов~В.\,В.
    	Задачу поставил:  Эксперт~И.\,О.
    	Консультант:  Консультант~И.\,О.}
    \email
    {varenik.nv@phystech.edu}
%\organization
 %   {$^1$Организация; $^2$Организация}
\abstract
    {Задача прогнозирования молекулярных свойств, например, биологической активности или растворимости на основе атомной структуры называется QSAR (Qualitative Structure Activity Relationships) предсказание. Это классическая задача в области разработки лекарств. Несмотря на то, что множество алгоритмов, таких как квантильная регрессия, нейронные сети на основе радиально-базисных функций являются приемлемыми решениями, все еще есть необходимость в более точной модели. В работе была выбрана модель сферических свёрточных нейронных сетей, изначально предложенная Taco S. Cohen et. al. для распознавания 3D-форм и положена под тщательное изучение в контексте QSAR предсказаний. Результаты модели сравниваются с результатами более общих моделей, таких как свёрточные нейронные сети, реккурентные нейронные сети, свёрточные нейронные сети с подающимся на вход графом и случайный лес. 
    

\bigskip
\textbf{Ключевые слова}: \emph {QSAR предсказание, сферические свёрточные нейронные сети, разработка лекарств}.}
\titleEng
    {JMLDA paper example: file jmlda-example.tex}
\authorEng
    {Author~F.\,S.$^1$, CoAuthor~F.\,S.$^2$, Name~F.\,S.$^2$}
\organizationEng
    {$^1$Organization; $^2$Organization}
\abstractEng
    {This document is an example of paper prepared with \LaTeXe\
    typesetting system and style file \texttt{jmlda.sty}.

    \bigskip
    \textbf{Keywords}: \emph{keyword, keyword, more keywords}.}
\begin{document}
\maketitle
%\linenumbers
\section{Введение}
Идея QSAR (Qualitative Structure Activity Relationships) заключается в том, чтобы связать 2D или 3D структурное представление молекулы с её биологическими или химическими свойствами. Эта задача очень важна в сфере разрабатывания лекарственных препаратов. Цель нашего исследования - создать точный инструмент прогнозирования QSAR. Было несколько попыток решить эту задачу. Изначально, было предложено использовать графическое представление молекулы для вычисления индекса Винера и терминального индекса Винера \cite{Wiener15}, которые коррелируют с такими понятиями как критическая точка \cite{Critical}, вязкость \cite{Visc}, но они не имеют четкой связи с растворимостью или активностью, которые особо важны в разработке лекарств. Машинное обучение, как развивающаяся наука дает возможность используя ее различные методы, такие как случайный лес, квантильная и самосогласованная регрессии, нейронные сети постепенно улучшать качество прогнозирования в различных отрослях задачи нахождения QSAR. Также рассмотривался вопрос рационального деления выборки на обучающую и тестовую \cite{journals/jcamd/GolbraikhT02}. Был сделан вывод, что оптимальный размер обучающей и тестовой выборки следует устанавливать на основе конкретного набора данных и типа используемых дескрипторов. Стоит отметить модель нейронных сетей, предложенную в 2014 году и активно используемую в наши дни, так как она дает довольно неплохие результаты \cite{journals/jcisd/ZakharovPSN14a}, в основе которой лежат радиальные базисные функции и самосогласованная регрессия.  
Метод, предложенный в данной статье основан на сферических свёрточных нейронных сетях \cite{SCNN}. Они обладают уникальной особенностью, такой как возможность проектирования сферического сигнала без искажений. Их разработчик Taco et. al. тестировал сферические свёртки в различных задачах, в том числе и для предсказания энергии распыления из молекулярной геометрии. Модель дала отличные результаты, поэтому возник интерес в её применении к задаче QSAR предсказаний. Основным недостатком предлагаемой модели является её сложность, связанная с большим числом ее параметров. Однако, ожидается, что данная модель станет универсальным решением нашей задачи. Результаты модели сравниваются с обычными моделями, такими как CNN, RNN, CNN с подающимся на вход графом и RF на данных, взятых из Benchmark Datasets. 

\bibliographystyle{plain}
\bibliography{Varenik2019Project47}
%\begin{thebibliography}{1}

%\bibitem{author09anyscience}
%    \BibAuthor{Author\;N.}
%    \BibTitle{Paper title}~//
%    \BibJournal{10-th Int'l. Conf. on Anyscience}, 2009.  Vol.\,11, No.\,1.  Pp.\,111--122.
%\bibitem{myHandbook}
%    \BibAuthor{Автор\;И.\,О.}
%    Название книги.
%    Город: Издательство, 2009. 314~с.
%\bibitem{author09first-word-of-the-title}
%    \BibAuthor{Автор\;И.\,О.}
%    \BibTitle{Название статьи}~//
%    \BibJournal{Название конференции или сборника},
%    Город:~Изд-во, 2009.  С.\,5--6.
%\bibitem{author-and-co2007}
%    \BibAuthor{Автор\;И.\,О., Соавтор\;И.\,О.}
%    \BibTitle{Название статьи}~//
%    \BibJournal{Название журнала}. 2007. Т.\,38, \No\,5. С.\,54--62.
%\bibitem{bibUsefulUrl}
%    \BibUrl{www.site.ru}~---
%    Название сайта.  2007.
%\bibitem{voron06latex}
%    \BibAuthor{Воронцов~К.\,В.}
%    \LaTeXe\ в~примерах.
%    2006.
%    \BibUrl{http://www.ccas.ru/voron/latex.html}.
%\bibitem{Lvovsky03}
%    \BibAuthor{Львовский~С.\,М.} Набор и вёрстка в пакете~\LaTeX.
%    3-е издание.
%    Москва:~МЦHМО, 2003.  448~с.
%\end{thebibliography}

% Решение Программного Комитета:
%\ACCEPTNOTE
%\AMENDNOTE
%\REJECTNOTE
\end{document}
